\documentclass[a4paper]{article}
\usepackage{hyperref} 
\usepackage[utf8]{inputenc}
\usepackage[english]{babel}
\usepackage{amsfonts}
\usepackage{amsthm}
\usepackage{amsmath}
\usepackage{enumerate}
 
\theoremstyle{definition}
\newtheorem{definition}{Definition}[section]
 
\theoremstyle{remark}
\newtheorem*{remark}{Remark}

\newcommand{\mdef}[2]{
	\theoremstyle{definition}
	\begin{definition}{#1}
	#2
	\end{definition}
}

\def \golf {Golf Problem Solver}

\renewcommand{\abstractname}{\golf}

\begin{document}

\title{\golf}
\author{Petr Geiger}
\date
\maketitle

\begin{abstract}
The goal of this project is to implement constrain satisfaction model
for general case of \href{http://mathworld.wolfram.com/SocialGolferProblem.html}
{Social Golfer Problem}. The model is implemented in \href{https://sicstus.sics.se/}
{SICStus Prolog} using its \href{https://sicstus.sics.se/sicstus/docs/latest4/html/sicstus.html/lib_002dclpfd.html#lib_002dclpfd}
{CSP over Finite Domains library}.
\end{abstract}

\section{Problem Definition}
From \href{http://mathworld.wolfram.com/SocialGolferProblem.html}{Wolfram Math World} original definition of problem is as follows:



\textit{``Twenty golfers wish to play in foursomes for 5 days. Is it possible for each golfer 
to play no more than once with any other golfer? The answer is yes, and the following table gives a solution.}

\begin{tabular}{| l || c | c | c | c | r |}
\hline
Mon & ABCD & EFGH & IJKL & MNOP	& QRST \\
\hline
Tue & AEIM & BJOQ & CHNT & DGLS	& FKPR \\
\hline
Wed & AGKO & BIPT & CFMS & DHJR	& ELNQ \\
\hline
Thu & AHLP & BKNS & CEOR & DFIQ	& GJMT \\
\hline
Fri & AFJN & BLMR& CGPQ	& DEKT & HIOS \\
\hline
\end{tabular} 

\textit{Event organizers for bowling, golf, bridge, or tennis frequently tackle problems of this sort, unaware of 
the problem complexity. In general, it is an unsolved problem. A table of known results is maintained by Harvey."
}


\subsection{General case problem definition}\label{general-def}

Lets first define problem parameters.

\mdef{}{ 
	\begin{enumerate}[(i)]
		\item $ N \in \mathbb{N} $ ... number of players 
		\item $ K \in \mathbb{N} $ ... size of groups
		\item $ D \in \mathbb{N} $ ... number of days
		\item $ G \in \mathbb{N} $ ... number of groups
	\end{enumerate}
}

Now when we have our problem parametrized we can describe model assuming those parameters.
But lets first define following simple notation.
\mdef{}{
	For given $S \in \mathbb{N}$ define:
	$$ [S] \equiv \{0,1,2,...,S-1\}.  $$
 }

\mdef{}{ Lets define groups as sets $\forall i \in [G], j \in [D]: S_{i,j}$ such  as:
	\begin{enumerate}[(i)]
		\item Players are identified by numbers in given groups. $$\forall p \in S_{i,j}: p \in [N]$$ 
		\item\label{groups-size} Every group has exactly $K$ players in it. $$ |S_{i,j}| = K$$  
		\item\label{one-player} Every player is at most once in group. $S_{i,j}$ are sets thus this holds trivaly. 
		\item\label{once-per-day} Every player is playing exactly once in every day.  $$ \forall i \in [D]: \bigsqcup_{ j \in [G]} S_{i,j} = [N] $$
		\item\label{player-pairs} Every player plays no more then once with any other player. Which is same as every pair of players 
		is in at most 1 group.  $$ \forall x,y \in [N]: |\{S | \exists i \in [D] \land \exists j \in [G]: x \in S_{i,j} \land y \in S_{i,j}\}| < 2 $$
	\end{enumerate}  
}

In following section our task is going to be to choose proper representation for this model in terms of CSP.



\section{CSP model representation}
% - v kazdem dni 
% (A) priradim kazdemu hraci skupinu
% vs
% (B) priradim kazdemu mistu ve skupine hrace
% vs
% (C) generovat rovnou inteligentne
% 
% (A)
% 5^20 k
% (1) podminka: kazda skupina ma prave k hrace
% 
% (B)
% 20^20
% (1a) podminka: hraci se ve skupine neopakuji
% (1b) podminka: kazdy hrac je pouze v jedne skupine 
% 
% 
% -pro vsechny skupiny mezi dny 
% (2) podminka: kazdy hrac hralje s kazdym nejvyse 1


Simpliest representation which comes right into our mind is using directly definition \ref{general-def}.

\subsection{Straight forward model}
We just create for each slot in each group in each day variable and will try to assign right players.

\mdef{}{ Lets define CSP model $M_a = \langle V_a,D_a,C_a \rangle$ such as:
	\begin{enumerate}[(i)]
		\item $$ V_a = \{V_k|  \forall i \in [D], \forall j \in [G], \forall p \in [K] \exists k \in \mathbb{N}: V_k \text{ is CSP variable } \land  k=i*G*K+j*K+p \} $$
		\item $$ D_a = \{D_i| \forall V_i \in V: D_i = [N] \} $$
		\item $$ C_a = \{ C_1, C_2, C_3, C_4\} $$
		\begin{enumerate}
			\item To ensure conditions \ref{groups-size} and \ref{one-player} we need constrain $C_1$ requiring that player can be assigned to at most one variable in certain group.
			\item To ensure condition \ref{once-per-day} we need constrain $C_2$ requiring that each player playes once per day. 
			\item To ensure condition \ref{player-pairs} we need constrain $C_3$ requiring that each pair of player plays at most once.
			\item Optionaly it would be good idea to create another constrain $C_4$ requiring order in variables of groups to prune state space.
		\end{enumerate}
	\end{enumerate}
}


One can see that although $M_a$ model is quite straight forward for variable representation,
we have to pay more attention for constrains. 

\subsubsection{Worst case estimate}

Lets make estimate how much states CSP solver has to go through in worst case.
Each variable $V_i$ can be assigned by $N$ domain values and there is 
$D*G*K$ of such variables thus estimate is:
$$N^{D*G*K} = 2^{\log_2{(N)}*D*G*K}$$

\subsubsection{Conclusion}
Problem of this model is obliviously unordered sets for group variables. 
This thought led us to following model.

\subsection{Model dealing with unordered sets}
This model increaces number of variables substantialy, but also reduces domains values substantialy.
\mdef{}{ Lets define CSP model $M_b = \langle V_b,D_b,C_b \rangle$ such as:
	\begin{enumerate}[(i)]
		\item $$ V_b = \{V_k|  \forall i \in [D], \forall j \in [G], \forall l \in [N] \exists k \in \mathbb{N}: V_k \text{ is CSP variable } \land  k=i*G*N+j*N+l \} $$
		\item $$ D_b = \{D_i| \forall V_i \in V: D_i = [2] \} $$
		\item $$ C_b = \{ C_1, C_2, C_3\} $$
		\begin{enumerate}
			\item To ensure condition \ref{groups-size} we need constrain $C_1$ ensuring there is going to be
			set exactly $K$ variables from each variables for given group.
			\item The condition \ref{one-player} is taken for granted in this model.
			\item To ensure condition \ref{once-per-day} we need constrain $C_2$ requiring that each player plays once per day. 
			\item To ensure condition \ref{player-pairs} we need constrain $C_3$ requiring that each pair of player plays at most once.
		\end{enumerate}
	\end{enumerate}
}

\subsubsection{Worst case estimate}

Lets make estimate how much states CSP solver has to go through in worst case.
Each variable $V_i$ can be assigned by $2$ domain values and there is 
$D*G*N$ of such variables thus estimate is:
$$2^{D*G*N}.$$

\subsubsection{Conclusion}
To sum this up we have increased number of variables in favor of decreasing number of domain variables.
We also need less constrains, because this model does not have problem with order of
variables in groups. This model also automaticly ensures that one player
can not be in group multiple times, which leads to simplification of constrains.
Lastly we have better worst case estimate.

Nevertheless, there is still quite a few constrains which just ensure that players
are assigned properly to groups. Following model thus try to look at problem from
different perspective in order to simplefy requried constrains.

\subsection{Dual model}
In this model there is met trade of between numbers of variables and numbers of values in domains.
We are going to assign groups to players which reduces number of variables significantly.

\mdef{}{ Lets define CSP model $M_c = \langle V_c,D_c,C_c \rangle$ such as:
	\begin{enumerate}[(i)]
		\item $$ V_c = \{V_k|  \forall i \in [D], \forall j \in [N] \exists k \in \mathbb{N}: V_k \text{ is CSP variable } \land  k=i*N+j \} $$
		\item $$ D_c = \{D_i| \forall V_i \in V: D_i = [G] \} $$
		\item $$ C_c = \{ C_1, C_2\} $$
		\begin{enumerate}
			\item To ensure condition \ref{groups-size} we need constrain $C_1$ ensuring that group assignment leds to 
			groups of equal size $K$.
			\item The condition \ref{one-player} is taken for granted in this model, because each player in each day has assigned only one group.
			\item The condition \ref{once-per-day} is also taken for granted, because every player  in each day has assigned group.
			\item We still need to ensure condition \ref{player-pairs} with constrain $C_2$ requiring that each pair of player plays at most once.
			\item Optionaly we can add constrain $C_3$ reducing only solutions where days are accepted only in lexically ascending order - which reduces symetries.
		\end{enumerate}
	\end{enumerate}
}

\subsubsection{Worst case estimate}
For each day we assign groups to  $N$ players, which makes following estimate:

$$ K^{D*N} = 2^{D*N*\log_2(K)}.$$

Lets find relation between $N$ and $K$ in favor for this model.

Since  $G*K = N$ because of condition \ref{once-per-day}, $M_b$ estimate can be
rewritten as follows:
$$ 2^{\frac{D*N^2}{K}}. $$ 

Now we want $M_c$ estimate to be lower than $M_b$ estimate:
$$ 2^{D*N*\log_2(K)} < 2^{\frac{D*N^2}{K}}.  $$
Logarithming both sides and reducing same elements on both sides of equtaion
leads to:
$$D*N*\log_2(K) < \frac{D*N^2}{K}  $$
$$\log_2(K) < \frac{N}{K}  $$
$$K*\log_2(K) < N. $$

Which gives us lower bound for $N$ such as:
$$ N \in \Omega(K\log_2 K).$$

\subsubsection{Conclusion}

Thus this model is more suitable for $N \in \Omega(K\log_2 K)$, but
which is quite natural for this problem. There has to be enough big $N$
to put players to groups of size $K$ anyway, which is rather not going to be big.

Because of relative simplicity of model and its good worst case performace we have
decided to implement this approach.

\section{CSP model $M_c$ implementation}
...

\section{Conclusion}

\end{document}
