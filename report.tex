\documentclass[a4paper]{article}
%\usepackage{url}
\usepackage{hyperref} 

\def \golf {Golf Problem Solver}

\renewcommand{\abstractname}{\golf}

\begin{document}

\title{\golf}
\author{Petr Geiger}
\date
\maketitle

\begin{abstract}
The goal of this project is to implement constrain satisfaction model
for general case of \href{http://mathworld.wolfram.com/SocialGolferProblem.html}
{Social Golfer Problem}. The model is implemented in \href{https://sicstus.sics.se/}
{SICStus Prolog} using its \href{https://sicstus.sics.se/sicstus/docs/latest4/html/sicstus.html/lib_002dclpfd.html#lib_002dclpfd}
{CSP over Finite Domains library}.
\end{abstract}

\section{Problem Definition}
Twenty golfers wish to play in foursomes for 5 days. Is it possible for each golfer to play no more than once with any other golfer? The answer is yes, and the following table gives a solution.
Event organizers for bowling, golf, bridge, or tennis frequently tackle problems of this sort, unaware of the problem complexity. In general, it is an unsolved problem. A table of known results is maintained by Harvey.
\section{Solution}
\subsection{CSP Model}
\subsubsection{Variables}
\subsubsection{Domain}
\subsubsection{Constrains}

\paragraph{Variables}
\subparagraph{Variables}
\subsection{}
\section{Test}
\section{Conclusion}

\end{document}
