\documentclass[a4paper]{article}
\usepackage{hyperref} 
\usepackage[utf8]{inputenc}
\usepackage[english]{babel}
\usepackage{amsfonts}
\usepackage{amsthm}
\usepackage{enumerate}
 
\theoremstyle{definition}
\newtheorem{definition}{Definition}[section]
 
\theoremstyle{remark}
\newtheorem*{remark}{Remark}

\newcommand{\mdef}[2]{
	\theoremstyle{definition}
	\begin{definition}{#1}
	#2
	\end{definition}
}

\def \golf {Golf Problem Solver}

\renewcommand{\abstractname}{\golf}

\begin{document}

\title{\golf}
\author{Petr Geiger}
\date
\maketitle

\begin{abstract}
The goal of this project is to implement constrain satisfaction model
for general case of \href{http://mathworld.wolfram.com/SocialGolferProblem.html}
{Social Golfer Problem}. The model is implemented in \href{https://sicstus.sics.se/}
{SICStus Prolog} using its \href{https://sicstus.sics.se/sicstus/docs/latest4/html/sicstus.html/lib_002dclpfd.html#lib_002dclpfd}
{CSP over Finite Domains library}.
\end{abstract}

\section{Problem Definition}
From \href{http://mathworld.wolfram.com/SocialGolferProblem.html}{Wolfram Math World} original definition of problem is as follows:



\textit{``Twenty golfers wish to play in foursomes for 5 days. Is it possible for each golfer 
to play no more than once with any other golfer? The answer is yes, and the following table gives a solution.}

\begin{tabular}{| l || c | c | c | c | r |}
\hline
Mon & ABCD & EFGH & IJKL & MNOP	& QRST \\
\hline
Tue & AEIM & BJOQ & CHNT & DGLS	& FKPR \\
\hline
Wed & AGKO & BIPT & CFMS & DHJR	& ELNQ \\
\hline
Thu & AHLP & BKNS & CEOR & DFIQ	& GJMT \\
\hline
Fri & AFJN & BLMR& CGPQ	& DEKT & HIOS \\
\hline
\end{tabular} 

\textit{Event organizers for bowling, golf, bridge, or tennis frequently tackle problems of this sort, unaware of 
the problem complexity. In general, it is an unsolved problem. A table of known results is maintained by Harvey."
}


\subsection{General case problem definition}

Lets first define problem parameters.

\mdef{}{ 
	\begin{enumerate}[(i)]
		\item $ N \in \mathbb{N} $ ... number of players 
		\item $ K \in \mathbb{N} $ ... size of groups
		\item $ D \in \mathbb{N} $ ... number of days
		\item $ G \in \mathbb{N} $ ... number of groups
	\end{enumerate}
}

Now when we have our problem parametrized we can describe model assuming those parameters.
But lets first define following simple notation.
\mdef{}{
	For given $S \in \mathbb{N}$ define:
	$$ [S] \equiv \{1,2,...S\}.  $$
 }

\mdef{}{ Lets define groups as sets $\forall i \in [G], j \in [D]: S_{i,j}$ such  as:
	\begin{enumerate}[(i)]
		\item Players are identified by numbers in given groups. $$\forall p \in S_{i,j}: p \in [N]$$ 
		\item Every group has exactly $K$ players in it. $$ |S_{i,j}| = K$$  
		\item Every player is at most once in group. $S_{i,j}$ are sets thus this holds trivaly. 
		\item Every player is playing exactly once in every day.  $$ \forall i \in [D]: \bigsqcup_{ j \in [G]} S_{i,j} = [N] $$
		\item Every player plays no more then once with any other player. Which is same as every pair of players 
		is in at most 1 group.  $$ \forall x,y \in [N]: |\{S | \exists i \in [D] \land \exists j \in [G]: x \in S_{i,j} \land y \in S_{i,j}\}| < 2 $$
	\end{enumerate}  
}

In following section our task is going to be to choose proper representation for this model in terms of CSP.



\section{CSP model representation}
% - v kazdem dni 
% (A) priradim kazdemu hraci skupinu
% vs
% (B) priradim kazdemu mistu ve skupine hrace
% vs
% (C) generovat rovnou inteligentne
% 
% (A)
% 5^20 k
% (1) podminka: kazda skupina ma prave k hrace
% 
% (B)
% 20^20
% (1a) podminka: hraci se ve skupine neopakuji
% (1b) podminka: kazdy hrac je pouze v jedne skupine 
% 
% 
% -pro vsechny skupiny mezi dny 
% (2) podminka: kazdy hrac hralje s kazdym nejvyse 1




\subsection{CSP Model}
\subsubsection{Variables}
\subsubsection{Domain}
\subsubsection{Constrains}

\paragraph{Variables}
\subparagraph{Variables}
\subsection{}
\section{Test}
\section{Conclusion}

\end{document}
